\documentclass{manual}
\usepackage[T1]{fontenc}

\title{cx\_Oracle}

\author{Anthony Tuininga}
\authoraddress{
        \strong{Colt Engineering}\\
        Email: \email{anthony.tuininga@gmail.com}
}

\date{\today}                   % date of release
\release{HEAD}                  % software release
\setreleaseinfo{}               % empty for final release
\setshortversion{HEAD}          % major.minor only for software

\begin{document}

\maketitle

\ifhtml
\chapter*{Front Matter\label{front}}
\fi

Copyright \copyright{} 2007 Colt Engineering. All rights reserved.\break
Copyright \copyright{} 2001-2007 Computronix. All rights reserved.

See the end of this document for complete license and permissions
information.

\begin{abstract}

\noindent
cx_Oracle is a Python extension module that allows access to Oracle and
conforms to the Python database API 2.0 specifications with a few exceptions.
See \url{http://www.python.org/topics/database/DatabaseAPI-2.0.html} for more
information on the Python database API specification.

\end{abstract}

\tableofcontents

\chapter{Module Interface\label{module}}

\begin{funcdesc}{Binary}{\var{string}}
  Construct an object holding a binary (long) string value.
\end{funcdesc}

\begin{funcdesc}{clientversion}{}
  Return the version of the client library being used as a 5 tuple. The five
  values are the major version, minor version, update number, patch number
  and port update number.

  \strong{NOTE:} This method is an extension to the DB API definition and is
  only available in Oracle 10g Release 2 and higher.
\end{funcdesc}

\begin{funcdesc}{Connection}{\optional{\var{user}, \var{password},
    \var{dsn}, \var{mode}, \var{handle}, \var{pool}, \var{threaded},
    \var{twophase}}}
\funcline{connect}{\optional{\var{user}, \var{password},
    \var{dsn}, \var{mode}, \var{handle}, \var{pool}, \var{threaded},
    \var{twophase}}}
  Constructor for creating a connection to the database. Return a Connection
  object (\ref{connobj}). All arguments are optional and can be specified as
  keyword parameters. The dsn (data source name) is the TNS entry (from the
  Oracle names server or tnsnames.ora file) or is a string like the one
  returned from makedsn(). If only one parameter is passed, a connect string is
  assumed which is to be of the format "user/password@dsn", the same format
  accepted by Oracle applications such as SQL*Plus. If the mode is specified,
  it must be one of SYSDBA or SYSOPER which are defined at the module level;
  otherwise it defaults to the normal mode of connecting. If the handle is
  specified, it must be of type OCISvcCtx* and is only of use when embedding
  Python in an application (like PowerBuilder) which has already made the
  connection. The pool is only valid in Oracle 9i and is a session pool object
  (\ref{sesspool}) which is the equivalent of calling pool.acquire(). The
  threaded attribute is expected to be a boolean expression which indicates
  whether or not Oracle should use the mode OCI_THREADED to wrap accesses to
  connections with a mutex. Doing so in single threaded applications imposes
  a performance penalty of about 10-15\% which is why the default is False.
  The twophase attribute is expected to be a boolean expression which indicates
  whether or not the attributes should be set on the connection object to allow
  for two phase commit. The default for this value is also False because of
  bugs in Oracle prior to Oracle 10g.
\end{funcdesc}

\begin{funcdesc}{Cursor}{\var{connection}}
  Constructor for creating a cursor.  Return a new Cursor object
  (\ref{cursorobj}) using the connection.

  \strong{NOTE:} This method is an extension to the DB API definition.
\end{funcdesc}

\begin{funcdesc}{Date}{\var{year}, \var{month}, \var{day}}
  Construct an object holding a date value.
\end{funcdesc}

\begin{funcdesc}{DateFromTicks}{\var{ticks}}
  Construct an object holding a date value from the given ticks value (number
  of seconds since the epoch; see the documentation of the standard Python
  time module for details).
\end{funcdesc}

\begin{funcdesc}{makedsn}{\var{host}, \var{port}, \var{sid}}
  Return a string suitable for use as the dsn for the connect() method. This
  string is identical to the strings that are defined by the Oracle names
  server or defined in the tnsnames.ora file.

  \strong{NOTE:} This method is an extension to the DB API definition.
\end{funcdesc}

\begin{funcdesc}{SessionPool}{\var{user}, \var{password}, \var{database},
        \var{min}, \var{max}, \var{increment}, \optional{\var{connectiontype},
        \var{threaded}, \var{getmode=cx_Oracle.SPOOL_ATTRVAL_NOWAIT}}}
  Create a session pool (see Oracle 9i documentation for more information)
  and return a session pool object (\ref{sesspool}). This allows for very
  fast connections to the database and is of primary use in a server where
  the same connection is being made multiple times in rapid succession (a
  web server, for example). If the connection type is specified, all calls to
  acquire() will create connection objects of that type, rather than the base
  type defined at the module level. The threaded attribute is expected to be a
  boolean expression which indicates whether or not Oracle should use the mode
  OCI_THREADED to wrap accesses to connections with a mutex. Doing so in single
  threaded applications imposes a performance penalty of about 10-15\% which is
  why the default is False.

  \strong{NOTE:} This method is an extension to the DB API definition and is
  only available in Oracle 9i.
\end{funcdesc}

\begin{funcdesc}{Time}{\var{hour}, \var{minute}, \var{second}}
  Construct an object holding a time value.
\end{funcdesc}

\begin{funcdesc}{TimeFromTicks}{\var{ticks}}
  Construct an object holding a time value from the given ticks value (number
  of seconds since the epoch; see the documentation of the standard Python
  time module for details).
\end{funcdesc}

\begin{funcdesc}{Timestamp}{\var{year}, \var{month}, \var{day},
    \var{hour}, \var{minute}, \var{second}}
  Construct an object holding a time stamp value.
\end{funcdesc}

\begin{funcdesc}{TimestampFromTicks}{\var{ticks}}
  Construct an object holding a time stamp value from the given ticks value
  (number of seconds since the epoch; see the documentation of the standard
  Python time module for details).
\end{funcdesc}

\section{Constants}

\begin{datadesc}{apilevel}
  String constant stating the supported DB API level. Currently '2.0'.
\end{datadesc}

\begin{datadesc}{buildtime}
  String constant stating the time when the binary was built.

  \strong{NOTE:} This attribute is an extension to the DB API definition.
\end{datadesc}

\begin{datadesc}{BINARY}
  This type object is used to describe columns in a database that are binary
  (in Oracle this is RAW columns).
\end{datadesc}

\begin{datadesc}{BFILE}
  This type object is used to describe columns in a database that are BFILEs.

  \strong{NOTE:} This attribute is an extension to the DB API definition.
\end{datadesc}

\begin{datadesc}{BLOB}
  This type object is used to describe columns in a database that are BLOBs.

  \strong{NOTE:} This attribute is an extension to the DB API definition.
\end{datadesc}

\begin{datadesc}{CLOB}
  This type object is used to describe columns in a database that are CLOBs.

  \strong{NOTE:} This attribute is an extension to the DB API definition.
\end{datadesc}

\begin{datadesc}{CURSOR}
  This type object is used to describe columns in a database that are cursors
  (in PL/SQL these are known as ref cursors).

  \strong{NOTE:} This attribute is an extension to the DB API definition.
\end{datadesc}

\begin{datadesc}{DATETIME}
  This type object is used to describe columns in a database that are dates.
\end{datadesc}

\begin{datadesc}{DBSHUTDOWN_ABORT}
  This constant is used in database shutdown to indicate that the program
  should not wait for current calls to complete or for users to disconnect from
  the database. Use only in unusual circumstances since database recovery may
  be necessary upon next startup.

  \strong{NOTE:} This attribute is an extension to the DB API definition.
\end{datadesc}

\begin{datadesc}{DBSHUTDOWN_FINAL}
  This constant is used in database shutdown to indicate that the instance can
  be truly halted. This should only be done after the database has been shut
  down in one of the other modes (except abort) and the database has been
  closed and dismounted using the appropriate SQL commands.

  \strong{NOTE:} This attribute is an extension to the DB API definition.
\end{datadesc}

\begin{datadesc}{DBSHUTDOWN_IMMEDIATE}
  This constant is used in database shutdown to indicate that all uncommitted
  transactions should be rolled back and any connected users should be
  disconnected.

  \strong{NOTE:} This attribute is an extension to the DB API definition.
\end{datadesc}

\begin{datadesc}{DBSHUTDOWN_TRANSACTIONAL}
  This constant is used in database shutdown to indicate that further
  connections should be prohibited and no new transactions should be allowed.
  It then waits for active transactions to complete.

  \strong{NOTE:} This attribute is an extension to the DB API definition.
\end{datadesc}

\begin{datadesc}{DBSHUTDOWN_TRANSACTIONAL_LOCAL}
  This constant is used in database shutdown to indicate that further
  connections should be prohibited and no new transactions should be allowed.
  It then waits for only local active transactions to complete.

  \strong{NOTE:} This attribute is an extension to the DB API definition.
\end{datadesc}

\begin{datadesc}{FIXED_CHAR}
  This type object is used to describe columns in a database that are fixed
  length strings (in Oracle this is CHAR columns); these behave differently
  in Oracle than varchar2 so they are differentiated here even though the DB
  API does not differentiate them.

  \strong{NOTE:} This attribute is an extension to the DB API definition.
\end{datadesc}

\begin{datadesc}{FNCODE_BINDBYNAME}
  This constant is used to register callbacks on the OCIBindByName() function
  of the OCI.

  \strong{NOTE:} This attribute is an extension to the DB API definition.
\end{datadesc}

\begin{datadesc}{FNCODE_BINDBYPOS}
  This constant is used to register callbacks on the OCIBindByPos() function
  of the OCI.

  \strong{NOTE:} This attribute is an extension to the DB API definition.
\end{datadesc}

\begin{datadesc}{FNCODE_DEFINEBYPOS}
  This constant is used to register callbacks on the OCIDefineByPos() function
  of the OCI.

  \strong{NOTE:} This attribute is an extension to the DB API definition.
\end{datadesc}

\begin{datadesc}{FNCODE_STMTEXECUTE}
  This constant is used to register callbacks on the OCIStmtExecute() function
  of the OCI.

  \strong{NOTE:} This attribute is an extension to the DB API definition.
\end{datadesc}

\begin{datadesc}{FNCODE_STMTFETCH}
  This constant is used to register callbacks on the OCIStmtFetch() function
  of the OCI.

  \strong{NOTE:} This attribute is an extension to the DB API definition.
\end{datadesc}

\begin{datadesc}{FNCODE_STMTPREPARE}
  This constant is used to register callbacks on the OCIStmtPrepare() function
  of the OCI.

  \strong{NOTE:} This attribute is an extension to the DB API definition.
\end{datadesc}

\begin{datadesc}{LOB}
  This type object is the Python type of BLOB and CLOB data that is returned
  from cursors.

  \strong{NOTE:} This attribute is an extension to the DB API definition.
\end{datadesc}

\begin{datadesc}{LONG_BINARY}
  This type object is used to describe columns in a database that are long
  binary (in Oracle these are LONG RAW columns).

  \strong{NOTE:} This attribute is an extension to the DB API definition.
\end{datadesc}

\begin{datadesc}{LONG_STRING}
  This type object is used to describe columns in a database that are long
  strings (in Oracle these are LONG columns).

  \strong{NOTE:} This attribute is an extension to the DB API definition.
\end{datadesc}

\begin{datadesc}{NATIVE_FLOAT}
  This type object is used to describe columns in a database that are of type
  binary_double or binary_float and is only available in Oracle 10g.

  \strong{NOTE:} This attribute is an extension to the DB API definition.
\end{datadesc}

\begin{datadesc}{NCLOB}
  This type object is used to describe columns in a database that are NCLOBs.

  \strong{NOTE:} This attribute is an extension to the DB API definition.
\end{datadesc}

\begin{datadesc}{NUMBER}
  This type object is used to describe columns in a database that are numbers.
\end{datadesc}

\begin{datadesc}{OBJECT}
  This type object is used to describe columns in a database that are objects.

  \strong{NOTE:} This attribute is an extension to the DB API definition.
\end{datadesc}

\begin{datadesc}{paramstyle}
  String constant stating the type of parameter marker formatting expected by
  the interface. Currently 'named' as in 'where name = :name'.
\end{datadesc}

\begin{datadesc}{ROWID}
  This type object is used to describe the pseudo column "rowid".
\end{datadesc}

\begin{datadesc}{PRELIM_AUTH}
  This constant is used to define the preliminary authentication mode required
  for performing database startup and shutdown.

  \strong{NOTE:} This attribute is an extension to the DB API definition.
\end{datadesc}

\begin{datadesc}{SPOOL_ATTRVAL_FORCEGET}
  This constant is used to define the "get" mode on session pools and indicates
  that a new connection will be returned if there are no free sessions
  available in the pool.

  \strong{NOTE:} This attribute is an extension to the DB API definition.
\end{datadesc}

\begin{datadesc}{SPOOL_ATTRVAL_NOWAIT}
  This constant is used to define the "get" mode on session pools and indicates
  that an exception is raised if there are no free sessions available in the
  pool.

  \strong{NOTE:} This attribute is an extension to the DB API definition.
\end{datadesc}

\begin{datadesc}{SPOOL_ATTRVAL_WAIT}
  This constant is used to define the "get" mode on session pools and indicates
  that the acquisition of a connection waits until a session is freed if there
  are no free sessions available in the pool.

  \strong{NOTE:} This attribute is an extension to the DB API definition.
\end{datadesc}

\begin{datadesc}{STRING}
  This type object is used to describe columns in a database that are strings
  (in Oracle this is VARCHAR2 columns).
\end{datadesc}

\begin{datadesc}{SYSDBA}
  Value to be passed to the connect() method which indicates that SYSDBA
  access is to be acquired. See the Oracle documentation for more details.

  \strong{NOTE:} This attribute is an extension to the DB API definition.
\end{datadesc}

\begin{datadesc}{SYSOPER}
  Value to be passed to the connect() method which indicates that SYSOPER
  access is to be acquired. See the Oracle documentation for more details.

  \strong{NOTE:} This attribute is an extension to the DB API definition.
\end{datadesc}

\begin{datadesc}{threadsafety}
  Integer constant stating the level of thread safety that the interface
  supports. Currently 2, which means that threads may share the module and
  connections, but not cursors. Sharing means that a thread may use a resource
  without wrapping it using a mutex semaphore to implement resource locking.

  Note that in order to make use of multiple threads in a program which intends
  to connect and disconnect in different threads, the threaded argument to the
  Connection constructor must be a true value. See the comments on the
  Connection constructor for more information (\ref{module}).
\end{datadesc}

\begin{datadesc}{TIMESTAMP}
  This type object is used to describe columns in a database that are
  timestamps.

  \strong{NOTE:} This attribute is an extension to the DB API definition and
  is only available in Oracle 9i.
\end{datadesc}

\begin{datadesc}{UCBTYPE_ENTRY}
  This constant is used to register callbacks on entry to the function
  of the OCI. In other words, the callback will be called prior to the
  execution of the OCI function.

  \strong{NOTE:} This attribute is an extension to the DB API definition.
\end{datadesc}

\begin{datadesc}{UCBTYPE_EXIT}
  This constant is used to register callbacks on exit from the function
  of the OCI. In other words, the callback will be called after the execution
  of the OCI function.

  \strong{NOTE:} This attribute is an extension to the DB API definition.
\end{datadesc}

\begin{datadesc}{UCBTYPE_REPLACE}
  This constant is used to register callbacks that completely replace the
  call to the OCI function.

  \strong{NOTE:} This attribute is an extension to the DB API definition.
\end{datadesc}

\begin{datadesc}{version}
  String constant stating the version of the module. Currently '\version{}'.

  \strong{NOTE:} This attribute is an extension to the DB API definition.
\end{datadesc}

\section{Exceptions}

\begin{datadesc}{Warning}
  Exception raised for important warnings and defined by the DB API but not
  actually used by cx_Oracle.
\end{datadesc}

\begin{datadesc}{Error}
  Exception that is the base class of all other exceptions defined by
  cx_Oracle and is a subclass of the Python StandardError exception (defined in
  the module exceptions).
\end{datadesc}

\begin{datadesc}{InterfaceError}
  Exception raised for errors that are related to the database interface rather
  than the database itself. It is a subclass of Error.
\end{datadesc}

\begin{datadesc}{DatabaseError}
  Exception raised for errors that are related to the database. It is a
  subclass of Error.
\end{datadesc}

\begin{datadesc}{DataError}
  Exception raised for errors that are due to problems with the processed data.
  It is a subclass of DatabaseError.
\end{datadesc}

\begin{datadesc}{OperationalError}
  Exception raised for errors that are related to the operation of the database
  but are not necessarily under the control of the progammer. It is a
  subclass of DatabaseError.
\end{datadesc}

\begin{datadesc}{IntegrityError}
  Exception raised when the relational integrity of the database is affected.
  It is a subclass of DatabaseError.
\end{datadesc}
 
\begin{datadesc}{InternalError}
  Exception raised when the database encounters an internal error.
  It is a subclass of DatabaseError.
\end{datadesc}
 
\begin{datadesc}{ProgrammingError}
  Exception raised for programming errors. It is a subclass of DatabaseError.
\end{datadesc}
 
\begin{datadesc}{NotSupportedError}
  Exception raised when a method or database API was used which is not
  supported by the database. It is a subclass of DatabaseError.
\end{datadesc}
 
\chapter{Connection Objects\label{connobj}}

\strong{NOTE}: Any outstanding changes will be rolled back when the connection
object is destroyed or closed.

\begin{datadesc}{action}
  This write-only attribute sets the action column in the v\$session table and
  is only available in Oracle 10g.

  \strong{NOTE:} This attribute is an extension to the DB API definition.
\end{datadesc}

\begin{datadesc}{autocommit}
  This read-write attribute determines whether autocommit mode is on or off.

  \strong{NOTE:} This attribute is an extension to the DB API definition.
\end{datadesc}

\begin{funcdesc}{begin}{}
\funcline{begin}{\optional{\var{formatId}, \var{transactionId},
    \var{branchId}}}
  Explicitly begin a new transaction. Without parameters, this explicitly
  begins a local transaction; otherwise, this explicitly begins a distributed
  (global) transaction with the given parameters. See the Oracle documentation
  for more details.

  Note that in order to make use of global (distributed) transactions, the
  twophase argument to the Connection constructor must be a true value. See the
  comments on the Connection constructor for more information (\ref{module}).

  \strong{NOTE:} This method is an extension to the DB API definition.
\end{funcdesc}

\begin{funcdesc}{cancel}{}
  Cancel a long-running transaction. This is only effective on non-Windows
  platforms.
\end{funcdesc}

\begin{datadesc}{clientinfo}
  This write-only attribute sets the client_info column in the v\$session table
  and is only available in Oracle 10g.

  \strong{NOTE:} This attribute is an extension to the DB API definition.
\end{datadesc}

\begin{funcdesc}{close}{}
  Close the connection now, rather than whenever __del__ is called. The
  connection will be unusable from this point forward; an Error exception will
  be raised if any operation is attempted with the connection. The same applies
  to any cursor objects trying to use the connection.
\end{funcdesc}

\begin{funcdesc}{commit}{}
  Commit any pending transactions to the database.
\end{funcdesc}

\begin{funcdesc}{cursor}{}
  Return a new Cursor object (\ref{cursorobj}) using the connection.
\end{funcdesc}

\begin{datadesc}{dsn}
  This read-only attribute returns the TNS entry of the database to which a
  connection has been established.

  \strong{NOTE:} This attribute is an extension to the DB API definition.
\end{datadesc}

\begin{datadesc}{encoding}
  This read-only attribute returns the IANA character set name of the character
  set in use by the Oracle client.

  \strong{NOTE:} This attribute is an extension to the DB API definition.
\end{datadesc}

\begin{datadesc}{maxBytesPerCharacter}
  This read-only attribute returns the maximum number of bytes each character
  can use for the client character set.

  \strong{NOTE:} This attribute is an extension to the DB API definition.
\end{datadesc}

\begin{datadesc}{module}
  This write-only attribute sets the module column in the v\$session table and
  is only available in Oracle 10g.

  \strong{NOTE:} This attribute is an extension to the DB API definition.
\end{datadesc}

\begin{datadesc}{nencoding}
  This read-only attribute returns the IANA character set name of the national
  character set in use by the Oracle client.

  \strong{NOTE:} This attribute is an extension to the DB API definition.
\end{datadesc}

\begin{datadesc}{password}
  This read-only attribute returns the password of the user which established
  the connection to the database.

  \strong{NOTE:} This attribute is an extension to the DB API definition.
\end{datadesc}

\begin{funcdesc}{ping}{}
  Ping the server which can be used to test if the connection is still active.

  \strong{NOTE:} This method is an extension to the DB API definition and is
  only available in Oracle 10g R2 and higher.
\end{funcdesc}

\begin{funcdesc}{prepare}{}
  Prepare the distributed (global) transaction for commit.

  \strong{NOTE:} This method is an extension to the DB API definition.
\end{funcdesc}

\begin{funcdesc}{register}{\var{code}, \var{when}, \var{function}}
  Register the function as an OCI callback. The code is one of the function
  codes defined in the Oracle documentation of which the most common ones are
  defined as constants in this module. The when parameter is one of
  UCBTYPE_ENTRY, UCBTYPE_EXIT or UCBTYPE_REPLACE. The function is a Python
  function which will accept the parameters that the OCI function accepts,
  modified as needed to return Python objects that are of some use. Note that
  this is a highly experimental method and can cause cx_Oracle to crash if not
  used properly. In particular, the OCI does not provide sizing information to
  the callback so attempts to access a variable beyond the allocated size will
  crash cx_Oracle. Use with caution.

  \strong{NOTE:} This method is an extension to the DB API definition.
\end{funcdesc}

\begin{funcdesc}{rollback}{}
  Rollback any pending transactions.
\end{funcdesc}

\begin{funcdesc}{shutdown}{\optional{\var{mode}}}
  Shutdown the database. In order to do this the connection must connected as
  SYSDBA or SYSOPER. First shutdown using one of the DBSHUTDOWN constants
  defined in the constants (\ref{constants}) section. Next issue the SQL
  statements required to close the database ("alter database close normal")
  and dismount the database ("alter database dismount") followed by a second
  call to this method with the DBSHUTDOWN_FINAL mode.

  \strong{NOTE:} This method is an extension to the DB API definition and is
  only available in Oracle 10g R2 and higher.
\end{funcdesc}

\begin{funcdesc}{startup}{\var{force=False}, \var{restrict=False}}
  Startup the database. This is equivalent to the SQL*Plus command
  "startup nomount". The connection must be connected as SYSDBA or SYSOPER with
  the PRELIM_AUTH option specified for this to work. Once this method has
  completed, connect again without the PRELIM_AUTH option and issue the
  statements required to mount ("alter database mount") and open ("alter
  database open") the database.

  \strong{NOTE:} This method is an extension to the DB API definition and is
  only available in Oracle 10g R2 and higher.
\end{funcdesc}

\begin{datadesc}{stmtcachesize}
  This read-write attribute specifies the size of the statement cache. This
  value can make a significant difference in performance (up to 100x) if you
  have a small number of statements that you execute repeatedly.

  \strong{NOTE:} This attribute is an extension to the DB API definition.
\end{datadesc}

\begin{datadesc}{tnsentry}
  This read-only attribute returns the TNS entry of the database to which a
  connection has been established.

  \strong{NOTE:} This attribute is an extension to the DB API definition.
\end{datadesc}

\begin{funcdesc}{unregister}{\var{code}, \var{when}}
  Unregister the function as an OCI callback. The code is one of the function
  codes defined in the Oracle documentation of which the most common ones are
  defined as constants in this module. The when parameter is one of
  UCBTYPE_ENTRY, UCBTYPE_EXIT or UCBTYPE_REPLACE.

  \strong{NOTE:} This method is an extension to the DB API definition.
\end{funcdesc}

\begin{datadesc}{username}
  This read-only attribute returns the name of the user which established the
  connection to the database.

  \strong{NOTE:} This attribute is an extension to the DB API definition.
\end{datadesc}

\begin{datadesc}{version}
  This read-only attribute returns the version of the database to which a
  connection has been established.

  \strong{NOTE:} This attribute is an extension to the DB API definition.
\end{datadesc}

\chapter{Cursor Objects\label{cursorobj}}

\begin{datadesc}{arraysize}
  This read-write attribute specifies the number of rows to fetch at a time
  internally and is the default number of rows to fetch with the fetchmany()
  call. It defaults to 1 meaning to fetch a single row at a time. Note that
  this attribute can drastically affect the performance of a query since it
  directly affects the number of network round trips that need to be performed.
\end{datadesc}

\begin{datadesc}{bindarraysize}
  This read-write attribute specifies the number of rows to bind at a time and
  is used when creating variables via setinputsizes() or var(). It defaults to
  1 meaning to bind a single row at a time.

  \strong{NOTE:} The DB API definition does not define this attribute.
\end{datadesc}

\begin{funcdesc}{arrayvar}{\var{dataType}, \var{value}, \optional{\var{size}}}
  Create an array variable associated with the cursor of the given type and
  size and return a variable object (\ref{varobj}). The value is either an
  integer specifying the number of elements to allocate or it is a list and
  the number of elements allocated is drawn from the size of the list. If the
  value is a list, the variable is also set with the contents of the list. If
  the size is not specified and the type is a string or binary, 4000 bytes
  (maximum allowable by Oracle) is allocated. This is needed for passing arrays
  to PL/SQL (in cases where the list might be empty and the type cannot be
  determined automatically) or returning arrays from PL/SQL.

  \strong{NOTE:} The DB API definition does not define this method.
\end{funcdesc}

\begin{funcdesc}{bindnames}{}
  Return the list of bind variable names bound to the statement. Note that the
  statement must have been prepared first.

  \strong{NOTE:} The DB API definition does not define this method.
\end{funcdesc}

\begin{funcdesc}{callfunc}{\var{name}, \var{returnType},
      \optional{\var{parameters=[]}}}
  Call a function with the given name. The return type is specified in the
  same notation as is required by setinputsizes(). The sequence of parameters
  must contain one entry for each argument that the function expects.
  The result of the call is the return value of the function.
\end{funcdesc}

\begin{funcdesc}{callproc}{\var{name}, \optional{\var{parameters=[]}}}
  Call a procedure with the given name. The sequence of parameters must contain
  one entry for each argument that the procedure expects. The result of the
  call is a modified copy of the input sequence. Input parameters are left
  untouched; output and input/output parameters are replaced with possibly new
  values.
\end{funcdesc}

\begin{funcdesc}{close}{}
  Close the cursor now, rather than whenever __del__ is called. The cursor will
  be unusable from this point forward; an Error exception will be raised if any
  operation is attempted with the cursor.
\end{funcdesc}

\begin{datadesc}{connection}
  This read-only attribute returns a reference to the connection object on
  which the cursor was created.

  \strong{NOTE:} This attribute is an extension to the DB API definition but it
  is mentioned in PEP 249 as an optional extension.
\end{datadesc}

\begin{datadesc}{description}
  This read-only attribute is a sequence of 7-item sequences. Each of these
  sequences contains information describing one result column: (name, type,
  display_size, internal_size, precision, scale, null_ok). This attribute will
  be None for operations that do not return rows or if the cursor has not had
  an operation invoked via the execute() method yet.

  The type will be one of the type objects defined at the module level.
\end{datadesc}

\begin{funcdesc}{execute}{\var{statement}, \optional{\var{parameters}},
    \var{**keywordParameters}}
  Execute a statement against the database. Parameters may be passed as a
  dictionary or sequence or as keyword arguments. If the arguments are a
  dictionary, the values will be bound by name and if the arguments are a
  sequence the values will be bound by position.

  A reference to the statement will be retained by the cursor. If None or the
  same string object is passed in again, the cursor will execute that
  statement again without performing a prepare or rebinding and redefining.
  This is most effective for algorithms where the same statement is used, but
  different parameters are bound to it (many times).

  For maximum efficiency when reusing an statement, it is best to use the
  setinputsizes() method to specify the parameter types and sizes ahead of
  time; in particular, None is assumed to be a string of length 1 so any
  values that are later bound as numbers or dates will raise a TypeError
  exception.

  If the statement is a query, a list of variable objects (\ref{varobj}) will
  be returned corresponding to the list of variables into which data will be
  fetched with the fetchone(), fetchmany() and fetchall() methods; otherwise,
  None will be returned.

  \strong{NOTE:} The DB API definition does not define the return value of this
  method.
\end{funcdesc}

\begin{funcdesc}{executemany}{\var{statement}, \var{parameters}}
  Prepare a statement for execution against a database and then execute it
  against all parameter mappings or sequences found in the sequence parameters.
  The statement is managed in the same way as the execute() method manages it.
\end{funcdesc}

\begin{funcdesc}{executemanyprepared}{\var{numIters}}
  Execute the previously prepared and bound statement the given number of
  times. The variables that are bound must have already been set to their
  desired value before this call is made.  This method was designed for the
  case where optimal performance is required as it comes at the expense of
  compatibility with the DB API.

  \strong{NOTE:} The DB API definition does not define this method.
\end{funcdesc}

\begin{funcdesc}{fetchall}{}
  Fetch all (remaining) rows of a query result, returning them as a list of
  tuples. An empty list is returned if no more rows are available. Note that
  the cursor's arraysize attribute can affect the performance of this
  operation, as internally reads from the database are done in batches
  corresponding to the arraysize.

  An exception is raised if the previous call to execute() did not produce any
  result set or no call was issued yet.
\end{funcdesc}

\begin{funcdesc}{fetchmany}{\optional{\var{numRows=cursor.arraysize}}}
  Fetch the next set of rows of a query result, returning a list of tuples. An
  empty list is returned if no more rows are available. Note that the cursor's
  arraysize attribute can affect the performance of this operation.

  The number of rows to fetch is specified by the parameter. If it is not
  given, the cursor's arrysize attribute determines the number of rows to be
  fetched. If the number of rows available to be fetched is fewer than the
  amount requested, fewer rows will be returned.

  An exception is raised if the previous call to execute() did not produce any
  result set or no call was issued yet.
\end{funcdesc}

\begin{funcdesc}{fetchone}{}
  Fetch the next row of a query result set, returning a single tuple or None
  when no more data is available.

  An exception is raised if the previous call to execute() did not produce any
  result set or no call was issued yet.
\end{funcdesc}

\begin{funcdesc}{fetchraw}{\optional{\var{numRows=cursor.arraysize}}}
  Fetch the next set of rows of a query result into the internal buffers of the
  defined variables for the cursor. The number of rows actually fetched is
  returned.  This method was designed for the case where optimal performance is
  required as it comes at the expense of compatibility with the DB API.

  An exception is raised if the previous call to execute() did not produce any
  result set or no call was issued yet.

  \strong{NOTE:} The DB API definition does not define this method.
\end{funcdesc}

\begin{funcdesc}{__iter__}{}
  Returns the cursor itself to be used as an iterator.

  \strong{NOTE:} This method is an extension to the DB API definition but it
  is mentioned in PEP 249 as an optional extension.
\end{funcdesc}

\begin{funcdesc}{next}{}
  Fetch the next row of a query result set, using the same semantics as
  the method fetchone().

  \strong{NOTE:} This method is an extension to the DB API definition but it
  is mentioned in PEP 249 as an optional extension.
\end{funcdesc}

\begin{datadesc}{numbersAsStrings}
  This integer attribute defines whether or not numbers should be returned as
  strings rather than integers or floating point numbers. This is useful to get
  around the fact that Oracle floating point numbers have considerably greater
  precision than C floating point numbers and not require a change to the SQL
  being executed.

  \strong{NOTE:} The DB API definition does not define this attribute.
\end{datadesc}

\begin{funcdesc}{parse}{\var{statement}}
  This can be used to parse a statement without actually executing it (this
  step is done automatically by Oracle when a statement is executed).

  \strong{NOTE:} The DB API definition does not define this method.
\end{funcdesc}

\begin{funcdesc}{prepare}{\var{statement}, \optional{\var{tag}}}
  This can be used before a call to execute() to define the statement that will
  be executed. When this is done, the prepare phase will not be performed when
  the call to execute() is made with None or the same string object as the
  statement. If specified (Oracle 9i and higher) the statement will be
  returned to the statement cache with the given tag. See the Oracle
  documentation for more information about the statement cache.

  \strong{NOTE:} The DB API definition does not define this method.
\end{funcdesc}

\begin{datadesc}{rowcount}
  This read-only attribute specifies the number of rows that have currently
  been fetched from the cursor (for select statements) or that have been
  affected by the operation (for insert, update and delete statements).
\end{datadesc}

\begin{datadesc}{rowfactory}
  This read-write attribute specifies a method to call for each row that is
  retrieved from the database. Ordinarily a tuple is returned for each row but
  if this attribute is set, the method is called with the argument tuple that
  would normally be returned and the result of the method is returned instead.

  \strong{NOTE:} The DB API definition does not define this attribute.
\end{datadesc}

\begin{funcdesc}{setinputsizes}{\var{*args}, \var{**keywordArgs}}
  This can be used before a call to execute() to predefine memory areas for the
  operation's parameters. Each parameter should be a type object corresponding
  to the input that will be used or it should be an integer specifying the
  maximum length of a string parameter. Use keyword arguments when binding by
  name and positional arguments when binding by position. The singleton None
  can be used as a parameter when using positional arguments to indicate that
  no space should be reserved for that position.
\end{funcdesc}

\begin{funcdesc}{setoutputsize}{\var{size}, \optional{\var{column}}}
  This can be used before a call to execute() to predefine memory areas for the
  long columns that will be fetched. The column is specified as an index into
  the result sequence. Not specifying the column will set the default size for
  all large columns in the cursor.
\end{funcdesc}

\begin{datadesc}{statement}
  This read-only attribute provides the string object that was previously
  prepared with prepare() or executed with execute().

  \strong{NOTE:} The DB API definition does not define this attribute.
\end{datadesc}

\begin{funcdesc}{var}{\var{dataType}, \optional{\var{size}}}
  Create a variable associated with the cursor of the given type and size and
  return a variable object (\ref{varobj}). If the size is not specified and the
  type is a string or binary, 4000 bytes (maximum allowable by Oracle) is
  allocated; if the size is not specified and the type is a long string or long
  binary, 128KB is allocated. This method was designed for use with PL/SQL
  in/out variables where the length or type cannot be determined automatically
  from the Python object passed in.

  \strong{NOTE:} The DB API definition does not define this method.
\end{funcdesc}

\chapter{Variable Objects\label{varobj}}

\strong{NOTE:} The DB API definition does not define this object.

\begin{datadesc}{allocelems}
  This read-only attribute returns the number of elements allocated in an
  array, or the number of scalar items that can be fetched into the variable.
\end{datadesc}

\begin{funcdesc}{getvalue}{\optional{\var{pos=0}}}
  Return the value at the given position in the variable.
\end{funcdesc}

\begin{datadesc}{maxlength}
  This read-only attribute returns the maximum length of the variable.
\end{datadesc}

\begin{funcdesc}{setvalue}{\var{pos}, \var{value}}
  Set the value at the given position in the variable.
\end{funcdesc}

\chapter{SessionPool Objects\label{sesspool}}

\strong{NOTE}: This object is an extension the DB API and is only available in
Oracle 9i.

\begin{funcdesc}{acquire}{}
  Acquire a connection from the session pool and return a connection
  object (\ref{connobj}).
\end{funcdesc}

\begin{datadesc}{busy}
  This read-only attribute returns the number of sessions currently acquired.
\end{datadesc}

\begin{funcdesc}{drop}{\var{connection}}
  Drop the connection from the pool which is useful if the connection is no
  longer usable (such as when the session is killed).
\end{funcdesc}

\begin{datadesc}{dsn}
  This read-only attribute returns the TNS entry of the database to which a
  connection has been established.
\end{datadesc}

\begin{datadesc}{increment}
  This read-only attribute returns the number of sessions that will be
  established when additional sessions need to be created.
\end{datadesc}

\begin{datadesc}{max}
  This read-only attribute returns the maximum number of sessions that the
  session pool can control.
\end{datadesc}

\begin{datadesc}{min}
  This read-only attribute returns the number of sessions with which the
  session pool was created and the minimum number of sessions that will be
  controlled by the session pool.
\end{datadesc}

\begin{datadesc}{name}
  This read-only attribute returns the name assigned to the session pool by
  Oracle.
\end{datadesc}

\begin{datadesc}{opened}
  This read-only attribute returns the number of sessions currently opened by
  the session pool.
\end{datadesc}

\begin{datadesc}{password}
  This read-only attribute returns the password of the user which established
  the connection to the database.
\end{datadesc}

\begin{funcdesc}{release}{\var{connection}}
  Release the connection back to the pool. This will be done automatically as
  well if the connection object is garbage collected.
\end{funcdesc}

\begin{datadesc}{timeout}
  This read-write attribute indicates the time (in seconds) after which idle
  sessions will be terminated in order to maintain an optimum number of open
  sessions.
\end{datadesc}

\begin{datadesc}{tnsentry}
  This read-only attribute returns the TNS entry of the database to which a
  connection has been established.
\end{datadesc}

\begin{datadesc}{username}
  This read-only attribute returns the name of the user which established the
  connection to the database.
\end{datadesc}

\chapter{LOB Objects\label{lobobj}}

\strong{NOTE}: This object is an extension the DB API. It is returned whenever
Oracle CLOB, BLOB and BFILE columns are fetched.

\strong{NOTE}: Internally, Oracle uses LOB locators which are allocated based
on the cursor array size. Thus, it is important that the data in the LOB object
be manipulated before another internal fetch takes place. The safest way to do
this is to use the cursor as an iterator. In particular, do not use the
fetchall() method. The exception "LOB variable no longer valid after
subsequent fetch" will be raised if an attempt to access a LOB variable after
a subsequent fetch is detected.

\begin{funcdesc}{close}{}
  Close the LOB. Call this when writing is completed so that the indexes
  associated with the LOB can be updated.
\end{funcdesc}
\begin{funcdesc}{fileexists}{}
  Return a boolean indicating if the file referenced by the BFILE type LOB
  exists.
\end{funcdesc}

\begin{funcdesc}{getchunksize}{}
  Return the chunk size for the internal LOB. Reading and writing to the LOB
  in chunks of multiples of this size will improve performance.
\end{funcdesc}

\begin{funcdesc}{getfilename}{}
  Return a two-tuple consisting of the directory alias and file name for a
  BFILE type LOB.
\end{funcdesc}

\begin{funcdesc}{isopen}{}
  Return a boolean indicating if the LOB is opened.
\end{funcdesc}

\begin{funcdesc}{open}{}
  Open the LOB for writing. This will improve performance when writing to a LOB
  in chunks and there are functional or extensible indexes associated with the
  LOB.
\end{funcdesc}

\begin{funcdesc}{read}{\optional{\var{offset = 1}, \optional{\var{amount}}}}
  Return a portion (or all) of the data in the LOB object.
\end{funcdesc}

\begin{funcdesc}{setfilename}{\var{dirAlias}, \var{name}}
  Set the directory alias and name of the BFILE type LOB.
\end{funcdesc}

\begin{funcdesc}{size}{}
  Returns the size of the data in the LOB object.
\end{funcdesc}

\begin{funcdesc}{trim}{\optional{\var{newSize = 0}}}
  Trim the LOB to the new size.
\end{funcdesc}

\begin{funcdesc}{write}{\var{data}, \optional{\var{offset = 1}}}
  Write the data to the LOB object at the given offset. Note that if you want
  to make the LOB value smaller, you must use the trim() function.
\end{funcdesc}

\chapter{Date Objects\label{dateobj}}

\strong{NOTE}: This object is an extension the DB API. It is returned whenever
Oracle date and timestamp (in Oracle 9i) columns are fetched and whenever the
constructor methods (Date(), Time(), Timestamp()) are called.

\strong{NOTE}: As of Python 2.4 cx_Oracle returns the datetime objects from the
standard library datetime module instead of these objects.

\begin{datadesc}{year}
  This read-only attribute returns the year.
\end{datadesc}

\begin{datadesc}{month}
  This read-only attribute returns the month.
\end{datadesc}

\begin{datadesc}{day}
  This read-only attribute returns the day.
\end{datadesc}

\begin{datadesc}{hour}
  This read-only attribute returns the hour.
\end{datadesc}

\begin{datadesc}{minute}
  This read-only attribute returns the minute.
\end{datadesc}

\begin{datadesc}{second}
  This read-only attribute returns the second.
\end{datadesc}

\begin{datadesc}{fsecond}
  This read-only attribute returns the fractional second.
\end{datadesc}

\chapter{License}

\centerline{\strong{LICENSE AGREEMENT FOR CX\_ORACLE \version}}

Copyright \copyright{} 2007, Colt Engineering, Edmonton, Alberta, Canada.
All rights reserved.\break
Copyright \copyright{} 2001-2007, Computronix (Canada) Ltd., Edmonton, Alberta, Canada.
All rights reserved.

Redistribution and use in source and binary forms, with or without
modification, are permitted provided that the following conditions are met:

\begin{enumerate}
\item
    Redistributions of source code must retain the above copyright notice,
    this list of conditions, and the disclaimer that follows.

\item
    Redistributions in binary form must reproduce the above copyright
    notice, this list of conditions, and the following disclaimer in the
    documentation and/or other materials provided with the distribution.

\item
    Neither the name of Colt Engineering nor the names of its contributors may
    be used to endorse or promote products derived from this software
    without specific prior written permission.
\end{enumerate}

DISCLAIMER:
THIS SOFTWARE IS PROVIDED BY COLT ENGINEERING AND CONTRIBUTORS *AS IS*
AND ANY EXPRESS OR IMPLIED WARRANTIES, INCLUDING, BUT NOT LIMITED TO,
THE IMPLIED WARRANTIES OF MERCHANTABILITY AND FITNESS FOR A
PARTICULAR PURPOSE ARE DISCLAIMED. IN NO EVENT SHALL COLT ENGINEERING
OR CONTRIBUTORS BE LIABLE FOR ANY DIRECT, INDIRECT, INCIDENTAL,
SPECIAL, EXEMPLARY, OR CONSEQUENTIAL DAMAGES (INCLUDING, BUT NOT
LIMITED TO, PROCUREMENT OF SUBSTITUTE GOODS OR SERVICES; LOSS OF
USE, DATA, OR PROFITS; OR BUSINESS INTERRUPTION) HOWEVER CAUSED
AND ON ANY THEORY OF LIABILITY, WHETHER IN CONTRACT, STRICT LIABILITY,
OR TORT (INCLUDING NEGLIGENCE OR OTHERWISE) ARISING IN ANY WAY OUT
OF THE USE OF THIS SOFTWARE, EVEN IF ADVISED OF THE POSSIBILITY OF
SUCH DAMAGE.



\end{document}

